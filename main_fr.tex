\documentclass[8pt,landscape,a4paper]{article}
\usepackage[utf8]{inputenc}
\usepackage[french]{babel}
\usepackage[T1]{fontenc}
\usepackage{tikz}
\usetikzlibrary{shapes,positioning,arrows,fit,calc,graphs,graphs.standard}
\usepackage[nosf]{kpfonts}
\usepackage[t1]{sourcesanspro}
\usepackage{multicol}
\usepackage{wrapfig}
\usepackage[top=1cm,bottom=1cm,left=1cm,right=1cm]{geometry}
\usepackage[framemethod=tikz]{mdframed}
\usepackage{microtype}
\usepackage{pdfpages}
\usepackage{lipsum} 
\usepackage{array} 

\let\bar\overline

\definecolor{myblue}{rgb}{0.28,0.0,1.0}

\def\firstcircle{(0,0) circle (1.5cm)}
\def\secondcircle{(0:2cm) circle (1.5cm)}

\colorlet{circle edge}{myblue}
\colorlet{circle area}{myblue!5}

\tikzset{filled/.style={fill=circle area, draw=circle edge, thick},
    outline/.style={draw=circle edge, thick}}
    
\pgfdeclarelayer{background}
\pgfsetlayers{background,main}

% \everymath\expandafter{\the\everymath \color{myblue}}
% \everydisplay\expandafter{\the\everydisplay \color{myblue}}

\renewcommand{\baselinestretch}{.8}
\pagestyle{empty}

\global\mdfdefinestyle{header}{%
linecolor=gray,linewidth=1pt,%
leftmargin=0mm,rightmargin=0mm,skipbelow=0mm,skipabove=0mm,
}

\newcommand{\header}{
\begin{mdframed}[style=header]
\footnotesize
\sffamily
\textbf{Vim Cheatsheet}\\
{\scriptsize seb@homedruon.com}
\end{mdframed}
}

\makeatletter % Author: https://tex.stackexchange.com/questions/218587/how-to-set-one-header-for-each-page-using-multicols
\renewcommand{\section}{\@startsection{section}{1}{0mm}%
                                {.7ex}%
                                {.7ex}%x
                                {\color{myblue}\sffamily\small\bfseries}}
\renewcommand{\subsection}{\@startsection{subsection}{1}{0mm}%
                                {.7ex}%
                                {.7ex}%x
                                {\sffamily\bfseries}}

\def\multi@column@out{%
   \ifnum\outputpenalty <-\@M
   \speci@ls \else
   \ifvoid\colbreak@box\else
     \mult@info\@ne{Re-adding forced
               break(s) for splitting}%
     \setbox\@cclv\vbox{%
        \unvbox\colbreak@box
        \penalty-\@Mv\unvbox\@cclv}%
   \fi
   \splittopskip\topskip
   \splitmaxdepth\maxdepth
   \dimen@\@colroom
   \divide\skip\footins\col@number
   \ifvoid\footins \else
      \leave@mult@footins
   \fi
   \let\ifshr@kingsaved\ifshr@king
   \ifvbox \@kludgeins
     \advance \dimen@ -\ht\@kludgeins
     \ifdim \wd\@kludgeins>\z@
        \shr@nkingtrue
     \fi
   \fi
   \process@cols\mult@gfirstbox{%
%%%%% START CHANGE
\ifnum\count@=\numexpr\mult@rightbox+2\relax
          \setbox\count@\vsplit\@cclv to \dimexpr \dimen@-1cm\relax
\setbox\count@\vbox to \dimen@{\vbox to 1cm{\header}\unvbox\count@\vss}%
\else
      \setbox\count@\vsplit\@cclv to \dimen@
\fi
%%%%% END CHANGE
            \set@keptmarks
            \setbox\count@
                 \vbox to\dimen@
                  {\unvbox\count@
                   \remove@discardable@items
                   \ifshr@nking\vfill\fi}%
           }%
   \setbox\mult@rightbox
       \vsplit\@cclv to\dimen@
   \set@keptmarks
   \setbox\mult@rightbox\vbox to\dimen@
          {\unvbox\mult@rightbox
           \remove@discardable@items
           \ifshr@nking\vfill\fi}%
   \let\ifshr@king\ifshr@kingsaved
   \ifvoid\@cclv \else
       \unvbox\@cclv
       \ifnum\outputpenalty=\@M
       \else
          \penalty\outputpenalty
       \fi
       \ifvoid\footins\else
         \PackageWarning{multicol}%
          {I moved some lines to
           the next page.\MessageBreak
           Footnotes on page
           \thepage\space might be wrong}%
       \fi
       \ifnum \c@tracingmulticols>\thr@@
                    \hrule\allowbreak \fi
   \fi
   \ifx\@empty\kept@firstmark
      \let\firstmark\kept@topmark
      \let\botmark\kept@topmark
   \else
      \let\firstmark\kept@firstmark
      \let\botmark\kept@botmark
   \fi
   \let\topmark\kept@topmark
   \mult@info\tw@
        {Use kept top mark:\MessageBreak
          \meaning\kept@topmark
         \MessageBreak
         Use kept first mark:\MessageBreak
          \meaning\kept@firstmark
        \MessageBreak
         Use kept bot mark:\MessageBreak
          \meaning\kept@botmark
        \MessageBreak
         Produce first mark:\MessageBreak
          \meaning\firstmark
        \MessageBreak
        Produce bot mark:\MessageBreak
          \meaning\botmark
         \@gobbletwo}%
   \setbox\@cclv\vbox{\unvbox\partial@page
                      \page@sofar}%
   \@makecol\@outputpage
     \global\let\kept@topmark\botmark
     \global\let\kept@firstmark\@empty
     \global\let\kept@botmark\@empty
     \mult@info\tw@
        {(Re)Init top mark:\MessageBreak
         \meaning\kept@topmark
         \@gobbletwo}%
   \global\@colroom\@colht
   \global \@mparbottom \z@
   \process@deferreds
   \@whilesw\if@fcolmade\fi{\@outputpage
      \global\@colroom\@colht
      \process@deferreds}%
   \mult@info\@ne
     {Colroom:\MessageBreak
      \the\@colht\space
              after float space removed
              = \the\@colroom \@gobble}%
    \set@mult@vsize \global
  \fi}

\makeatother
\setlength{\parindent}{0pt}

\begin{document}
\small

\begin{multicols*}{3}

    \section{Lancer vim ... et sortir}

    \subsection{invocation}

    \begin{tabular}{m{7cm} l}
        \texttt{vim file1 file2 ... fileN}   & \\
        \texttt{vim +linenumber file}  & \\
        \texttt{vim scp://login@serveur/path/to/file}  & \\
        \texttt{vim scp://login@serveur/path/}  & \\
    \end{tabular}

    \subsection{Sortir de vim}
    \begin{tabular}{m{2cm} l}
        \texttt{:q}   & Quitter\\
        \texttt{:wq}  & Sauver et quitter \\
        \texttt{:q!}  & Quitter sans sauver\\
    \end{tabular}

    \section{Se déplacer}
    \begin{tabular}{m{2cm} l}
        \texttt{h,j,k,l}  & Déplacer le curseur \\
        $\leftarrow$ $\downarrow$ $\uparrow$ $\rightarrow$  & Déplacer le curseur \\
    \end{tabular}
    
    \subsection{Dans la ligne}
    \begin{tabular}{m{2cm} l}
        \texttt{0}  & Sauter au premier caractère\\
        \texttt{\$} & Sauter au dernier caractère \\
        \texttt{\^} & Sauter au premier mot de la ligne\\
        \texttt{w}  & Mot suivant\\
        \texttt{b}  & Mot précédent\\
        \texttt{f char} & Sauter sur la prochaine occurrence d'un car. \\
        \texttt{t char} & Sauter avant la prochaine occurrence d'un car. \\
        \texttt{F char} & Sauter sur l'occurrence précédente d'un car. \\
        \texttt{T char} & Sauter après l'occurrence précédente d'un car. \\
        \texttt{;} & Répéter la dernière commande f/t \\
        \texttt{,} & Répéter la dernière commande F/T \\
    \end{tabular}
    
    \subsection{Dans le fichier }
    \begin{tabular}{m{2cm} l}
        \texttt{gg} & Sauter au début de la première ligne\\
        \texttt{G}  & Sauter au début de la dernière ligne\\
        \texttt{:n} & Sauter à la ligne n \\
        \texttt{\%} & Sauter à la parenthèse, accolade, etc. correspondante. \\
    \end{tabular}

    \subsection{Dans l'écran}
    \begin{tabular}{m{2cm} l}
        \texttt{zz} & Centrer la ligne courante \\
        \texttt{zb} & Envoyer la ligne courante en bas\\
        \texttt{zt} & Envoyer la ligne courante en haut\\
        \texttt{H} & Sauter en haut de l'écran\\
        \texttt{M} & Sauter au milieu de l'écran\\
        \texttt{L} & Sauter en bas de l'écran\\
        \texttt{CTRL + b} & Scroller vers le haut\\
        \texttt{CTRL + u} & Scroller vers le haut (demi écran)\\
        \texttt{CTRL + d} & Scroller vers le bas (demi écran)\\
        \texttt{CTRL + f} & Scroller vers le bas\\
    \end{tabular}

    \section{Passer en mode insertion}
    \begin{tabular}{m{2cm} l}
        \texttt{i}  & A la position du curseur\\
        \texttt{a}  & Après le curseur\\
        \texttt{A}  & A la fin de la ligne\\
        \texttt{o}  & Ouvrir une nouvelle ligne (après)\\
        \texttt{O}  & Ouvrir une nouvelle ligne (avant)\\
    \end{tabular}
    
    \section{Passer en mode visuel}
    \begin{tabular}{m{2cm} l}
        \texttt{v}& Mode visuel (caractère)\\
        \texttt{V}& Mode visuel (ligne)\\
        \texttt{CTRL+v}& Mode visuel(bloc)\\
    \end{tabular}

    \section{Mode normal}
    \begin{tabular}{m{2cm} l}
        \texttt{u}& Undo\\
        \texttt{CTRL+r}& Redo\\
        \texttt{.}& Répéter la dernière édition\\
    \end{tabular}\\

    \section{Sans quitter le mode insertion}
    \begin{tabular}{m{2cm} l}
        \texttt{CTRL+w} & Effacer le mot précédent\\
        \texttt{CTRL+u} & Effacer jusqu'au début de la ligne\\
        \texttt{CTRL+o} & Effectuer une seule commande\\
    \end{tabular}\\

    \section{Ouvrir / Sauver}
    \begin{tabular}{m{2cm} l}
        \texttt{:e fichier}& Ouvrir un fichier dans un buffer\\
        \texttt{:w}& Sauver\\
        \texttt{:w fichier}& Sauver sous\\
        \texttt{:browse old}& Fichiers récemment ouverts\\

    \end{tabular}\\
    \section{Buffers}
    \begin{tabular}{m{2cm} l}
        \texttt{:ls}& Lister tous les buffers\\
        \texttt{:b7}& Passer au buffer 7\\
        \texttt{:bn}& Buffer suivant\\
        \texttt{:bp}& Buffer précédent\\
    \end{tabular}\\

    \section{Fenêtres}
    \begin{tabular}{m{4cm} l}
        \texttt{CTRL+w v}& Split vertical\\
        \texttt{CTRL+w s}& Split horizontal\\
        \texttt{CTRL+w h,j,k,l }& Changer de fenêtre\\
        \texttt{CTRL+w SHIFT h,j,k,l }& Déplacer une fenêtre\\
        \texttt{CTRL+w c }&Fermer une fenêtre\\
    \end{tabular}\\

    Bien sûr, on peut substituer h,j,k,l par les flêches \\
        
    \begin{tabular}{m{3cm} l}
	\texttt{:sp file}& Ouvrir une fenêtre dans un split\\
    \end{tabular}\\
    
    \section{Sessions}
    \begin{tabular}{m{4cm} l}
        \texttt{:mks fichier.vim} & Sauvegarder la session\\
        \texttt{:source fichier.vim} & Restaurer la session\\
    \end{tabular}\\

    \section{Rechercher}
    \begin{tabular}{m{2cm} l}
        \texttt{/word}& Trouver la prochaine occurrence de "word"\\
        \texttt{?word}& Trouver la dernière occurrence de "word"\\
        \texttt{n}& Suivant\\
        \texttt{N}& Précédent\\
    \end{tabular}
    
    \section{ Macros }
    \begin{tabular}{m{2cm} l}
        \texttt{qa} & Enregistrer la macro a\\
        \texttt{q} & Arrêter l'enregistrement\\
        \texttt{@a} & Démarrer la macro a\\
        \texttt{@@} & Relancer la dernière macro\\
    \end{tabular}\\
    
    \section{Copier / Coller}
    \subsection{ Copier la sélection courante }
    \begin{tabular}{m{2cm} l}
        \texttt{y}& (yank) Copier\\
        \texttt{d}& (delete) Couper\\
    \end{tabular}
    \subsection{ Coller }
    \begin{tabular}{m{2cm} l}
        \texttt{p}& (put) Coller (après)\\
        \texttt{P}& (put) Coller (avant)\\
    \end{tabular}
    \subsection{ Copier sans selection }
    \begin{tabular}{m{2cm} l}
        \texttt{yy}& Copier la ligne courante\\
        \texttt{dd}& Couper la ligne courante\\
        \texttt{yny}& Copier n lignes\\
        \texttt{yw}& Copier le mot courant\\
        \texttt{y\$}& Copier du curseur à la fin de la ligne\\
    \end{tabular}

    \subsection{ Les registres }
    \begin{tabular}{m{2cm} l}
        \texttt{"ay}& Copier dans le registre a\\
        \texttt{"Ay}& Copiet et ajouter au register a\\
        \texttt{:reg}& Afficher les registres\\
    \end{tabular} \\

    Les registres utilisables sont les suivants \\

    \begin{tabular}{m{2cm} l}
        \texttt{a..z}& 26 registres d'usage général\\
        \texttt{1..9}& 9 derniers usages de d\\
        \texttt{+}   & Registre de copier-coller système (ctrl+c)\\
        \texttt{*}   & Registre de copier-coller système (souris)\\
        \texttt{\%}  & Nom de fichier\\
    \end{tabular} \\


    \section{Rechercher \& remplacer}
    
    \texttt{:\%s{\char`\/}expr1/expr2/g} \\
    
    \begin{tabular}{m{2cm} l}
        \texttt{expr1 }   &  Expression recherchée \\
        \texttt{expr2 }   &  Expression de remplacement\\
        \texttt{\% }   &  Portée\\
        \texttt{g }   &  Options\\
    \end{tabular}\\

    \subsection{Portée:}
    \begin{tabular}{m{2cm} l}
        \texttt{n,m}  & De la ligne n à m\\
        \texttt{.,+5} & Ligne courante et 5 suivantes \\
        \texttt{\%}   & Dans tout le fichier\\
    \end{tabular}\\

    \subsection{Options:}
    \begin{tabular}{m{2cm} l}
        \texttt{g} & Continuer après le premier\\
        \texttt{c} & Demander avant de remplacer\\
        \texttt{i} & Ignorer la casse\\
    \end{tabular}\\
    
    \subsection{Replace with foobar (y/n/a/q/l/\^{}E/\^{}Y) ?}
    \begin{tabular}{m{2cm} l}
        \texttt{y} & Oui\\
        \texttt{n} & Non\\
        \texttt{a} & Replacer tous les restants\\
        \texttt{q} & Quitter\\
        \texttt{l} & Dernier !\\
        \texttt{CTRL + e} & Scroller vers le haut\\
        \texttt{CTRL + y} & Scroller vers le bas\\
    \end{tabular}\\
    
    \section{ Marquer des lignes }
    \begin{tabular}{m{2cm} l}
        \texttt{m a} & Marquer la ligne dans le registre a\\
        \texttt{' a} & Retourner à la marque a\\
    \end{tabular}\\
    
    \section{Commandes et shell}
    \begin{tabular}{m{2cm} l}
        \texttt{:r file} & Inserer le contenu de <file> \\
        \texttt{:! cmd}  & Execute la commande cmd dans un shell\\
        \texttt{:r! cmd} & Execute la commande cmd et insère la réponse \\
        \texttt{:term} & Ouvrir un terminal\\
    \end{tabular}\\

    \section{ Tabs }
    \begin{tabular}{m{2cm} l}
        \texttt{:tabnew} & Ouvrir un tab\\
        \texttt{gt} & Aller au tab suivant\\
        \texttt{gT} & Aller au tab précédent\\
        \texttt{:tab ba} & Editer tous les buffers dans des tabs\\
    \end{tabular}\\
    
    \section{Correction orthographique}
    \begin{tabular}{m{4cm} l}
        \texttt{:spell lang=fr} & Démarrer la correction orthographique\\
        \texttt{:setlocal spell lang=fr} & Correction dans le buffer courant\\
        \texttt{:set nospell} & Arrêter la correction orthographique\\
        \texttt{]s} & Prochain mot erroné\\
        \texttt{[s} & Mot erroné précédent\\
        \texttt{z=} & Suggerer une correction\\
        \texttt{zg} & Ajouter au dictionnaire\\
        \texttt{zw} & Marque un mot comme incorrect\\
    \end{tabular}\\

    \section{Le coin du codeur}
    \subsection{Incrémenter / Décrémenter}
    \begin{tabular}{m{2cm} l}
        \texttt{CTRL+a} & Ajouter 1 à un nombre\\
        \texttt{CTRL+x} & Soustraire 1 à un nombre\\
    \end{tabular}\\
    \subsection{Indentation}
    \begin{tabular}{m{2cm} l}
        \texttt{=}  & Indentation de la selection\\
        \texttt{==} & Indentation de la ligne \\
        \texttt{$>$}  & Augmenter l'indentation \\
        \texttt{$<$}  & Réduire l'indentation \\
        \texttt{$>>$} & Augmenter l'indentation de la ligne courante\\
        \texttt{$<<$} & Diminuer l'indentation de la ligne courante\\
        \texttt{$n>>$} & Augmente l'indentation de n lignes\\
    \end{tabular}\\
    \subsection{Compilation}
    \begin{tabular}{m{2cm} l}
        \texttt{:make}  & Invoque make dans le répertoire courant\\
    \end{tabular}\\

    \section{Gestionnaire de fichier (netrw) }
    \begin{tabular}{m{2cm} l}
        \texttt{:Explore} & Ouvrir le gestionnaire de fichier\\
    \end{tabular}\\
    \subsection{ Operations}
    \begin{tabular}{m{2cm} l}
        \texttt{D} & Effacer fichier / répertoire\\
        \texttt{d} & Créer un nouveau répertoire\\
        \texttt{\%} & Ouvrir un nouveau fichier\\
    \end{tabular}\\
    \subsection{ Marquer / Démarquer}
    \begin{tabular}{m{2cm} l}
        \texttt{mf} & Marquer un fichier\\
        \texttt{mu} & Démarquer tous les fichiers\\
        \texttt{mp} & Afficher les fichiers marqués\\
        \texttt{mt} & Tagger le repertoire courant\\
    \end{tabular}\\
    \subsection{ Opérations sur les fichiers marqués}
    \begin{tabular}{m{2cm} l}
        \texttt{me} & Editer\\
        \texttt{md} & Diff\\
        \texttt{mc} & Copier dans le repertoire taggué\\
        \texttt{mm} & Déplacer dans le repertoire taggué\\
    \end{tabular}\\

    \section{Gestion des fenëtres }

    \subsection{Scroll simultané }

    \begin{tabular}{m{3cm} l}
        \texttt{:set scrollbind} & Sycnhroniser le défilement\\
    \end{tabular}\\

    \subsection{Redimensionner les fenêtres}

    \begin{tabular}{m{2cm} l}
        \texttt{CTRL-w -} & Réduire la hauteur \\
        \texttt{CTRL-w +} & Augmenter la hauteur \\
        \texttt{CTRL-w <} & Réduire la largeur \\
        \texttt{CTRL-w >} & Augmenter la largeur \\
        \texttt{CTRL-w =} & Egaliser tous les splits \\
    \end{tabular}\\

    \section{Parlez-vous vim ?}

    \subsection{ Format }

    <Numbre de fois> + <Commande> + <Mouvement> ou <Text object>

    \subsection{Commandes}
    \begin{tabular}{m{2cm} l}
        \texttt{c} & Changer \\
        \texttt{d} & Effacer \\
        \texttt{y} & Coller \\
    \end{tabular}\\


    \subsection{Mouvement}
    \begin{tabular}{m{2cm} l}
        \texttt{)} & Déplacer le curseur à la fin de la place \\
        \texttt{(} & Déplacer le curseur au début de la phrase \\
        \texttt{\}} & Déplacer le curseur au prochain paragraphe \\
        \texttt{\{} & Déplacer le curseur au début du paragraphe précédent\\
        \texttt{w} & Déplacer le curseur au prochain mot \\
        \texttt{b} & Déplacer le curseur au mot précédent \\
        \texttt{\$} & Déplacer le curseur à la fin de la ligne logique\\
        \texttt{0} & Déplacer le curseur au début de la ligne logique\\
        \texttt{G} & Déplacer le curseur à la fin du fichier \\
        \texttt{gg} & Déplacer le curseur au début du fichier \\
    \end{tabular}\\

    \subsection{Text Objects}
    
    \begin{tabular}{m{2cm} l}
        \texttt{s} & Phrase \\
        \texttt{w} & Mot \\
        \texttt{p} & Paragraphe \\
        \texttt{a"} & Chaîne (guillements) \\
        \texttt{a'} & Chaîne (Single quote)\\
    \end{tabular}\\

    \subsection{Préciser les Text objects}

    \begin{tabular}{m{2cm} l}
        \texttt{w}  & Du curseur à la findu mot \\
        \texttt{iw} & Mot complet \\
        \texttt{aw} & Mot complet et espaces\\
    \end{tabular}\\
    
    \section{ Digraphs }
    \begin{tabular}{m{2cm} l}
        \texttt{CTRL-k code} & Insère un digraph\\
    \end{tabular}\\

%    \begin{tabular}{m{2cm} l}
%        \texttt{} &  \\
%    \end{tabular}



\end{multicols*}

\end{document}

