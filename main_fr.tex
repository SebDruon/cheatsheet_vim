\documentclass[8pt,landscape,a4paper]{article}
\usepackage[utf8]{inputenc}
\usepackage[french]{babel}
\usepackage[T1]{fontenc}
\usepackage{tikz}
\usetikzlibrary{shapes,positioning,arrows,fit,calc,graphs,graphs.standard}
\usepackage[nosf]{kpfonts}
\usepackage[t1]{sourcesanspro}
\usepackage{multicol}
\usepackage{wrapfig}
\usepackage[top=1cm,bottom=1cm,left=1cm,right=1cm]{geometry}
\usepackage[framemethod=tikz]{mdframed}
\usepackage{microtype}
\usepackage{pdfpages}
\usepackage{lipsum} 
\usepackage{array} 

\let\bar\overline

\definecolor{myblue}{rgb}{0.28,0.0,1.0}

\def\firstcircle{(0,0) circle (1.5cm)}
\def\secondcircle{(0:2cm) circle (1.5cm)}

\colorlet{circle edge}{myblue}
\colorlet{circle area}{myblue!5}

\tikzset{filled/.style={fill=circle area, draw=circle edge, thick},
    outline/.style={draw=circle edge, thick}}
    
\pgfdeclarelayer{background}
\pgfsetlayers{background,main}

% \everymath\expandafter{\the\everymath \color{myblue}}
% \everydisplay\expandafter{\the\everydisplay \color{myblue}}

\renewcommand{\baselinestretch}{.8}
\pagestyle{empty}

\global\mdfdefinestyle{header}{%
linecolor=gray,linewidth=1pt,%
leftmargin=0mm,rightmargin=0mm,skipbelow=0mm,skipabove=0mm,
}

\newcommand{\header}{
\begin{mdframed}[style=header]
\footnotesize
\sffamily
\textbf{Aide mémoire Vim}\\
{\scriptsize seb@homedruon.com}
\end{mdframed}
}

\makeatletter % Author: https://tex.stackexchange.com/questions/218587/how-to-set-one-header-for-each-page-using-multicols
\renewcommand{\section}{\@startsection{section}{1}{0mm}%
                                {.7ex}%
                                {.7ex}%x
                                {\color{myblue}\sffamily\small\bfseries}}
\renewcommand{\subsection}{\@startsection{subsection}{1}{0mm}%
                                {.7ex}%
                                {.7ex}%x
                                {\sffamily\bfseries}}

\def\multi@column@out{%
   \ifnum\outputpenalty <-\@M
   \speci@ls \else
   \ifvoid\colbreak@box\else
     \mult@info\@ne{Re-adding forced
               break(s) for splitting}%
     \setbox\@cclv\vbox{%
        \unvbox\colbreak@box
        \penalty-\@Mv\unvbox\@cclv}%
   \fi
   \splittopskip\topskip
   \splitmaxdepth\maxdepth
   \dimen@\@colroom
   \divide\skip\footins\col@number
   \ifvoid\footins \else
      \leave@mult@footins
   \fi
   \let\ifshr@kingsaved\ifshr@king
   \ifvbox \@kludgeins
     \advance \dimen@ -\ht\@kludgeins
     \ifdim \wd\@kludgeins>\z@
        \shr@nkingtrue
     \fi
   \fi
   \process@cols\mult@gfirstbox{%
%%%%% START CHANGE
\ifnum\count@=\numexpr\mult@rightbox+2\relax
          \setbox\count@\vsplit\@cclv to \dimexpr \dimen@-1cm\relax
\setbox\count@\vbox to \dimen@{\vbox to 1cm{\header}\unvbox\count@\vss}%
\else
      \setbox\count@\vsplit\@cclv to \dimen@
\fi
%%%%% END CHANGE
            \set@keptmarks
            \setbox\count@
                 \vbox to\dimen@
                  {\unvbox\count@
                   \remove@discardable@items
                   \ifshr@nking\vfill\fi}%
           }%
   \setbox\mult@rightbox
       \vsplit\@cclv to\dimen@
   \set@keptmarks
   \setbox\mult@rightbox\vbox to\dimen@
          {\unvbox\mult@rightbox
           \remove@discardable@items
           \ifshr@nking\vfill\fi}%
   \let\ifshr@king\ifshr@kingsaved
   \ifvoid\@cclv \else
       \unvbox\@cclv
       \ifnum\outputpenalty=\@M
       \else
          \penalty\outputpenalty
       \fi
       \ifvoid\footins\else
         \PackageWarning{multicol}%
          {I moved some lines to
           the next page.\MessageBreak
           Footnotes on page
           \thepage\space might be wrong}%
       \fi
       \ifnum \c@tracingmulticols>\thr@@
                    \hrule\allowbreak \fi
   \fi
   \ifx\@empty\kept@firstmark
      \let\firstmark\kept@topmark
      \let\botmark\kept@topmark
   \else
      \let\firstmark\kept@firstmark
      \let\botmark\kept@botmark
   \fi
   \let\topmark\kept@topmark
   \mult@info\tw@
        {Use kept top mark:\MessageBreak
          \meaning\kept@topmark
         \MessageBreak
         Use kept first mark:\MessageBreak
          \meaning\kept@firstmark
        \MessageBreak
         Use kept bot mark:\MessageBreak
          \meaning\kept@botmark
        \MessageBreak
         Produce first mark:\MessageBreak
          \meaning\firstmark
        \MessageBreak
        Produce bot mark:\MessageBreak
          \meaning\botmark
         \@gobbletwo}%
   \setbox\@cclv\vbox{\unvbox\partial@page
                      \page@sofar}%
   \@makecol\@outputpage
     \global\let\kept@topmark\botmark
     \global\let\kept@firstmark\@empty
     \global\let\kept@botmark\@empty
     \mult@info\tw@
        {(Re)Init top mark:\MessageBreak
         \meaning\kept@topmark
         \@gobbletwo}%
   \global\@colroom\@colht
   \global \@mparbottom \z@
   \process@deferreds
   \@whilesw\if@fcolmade\fi{\@outputpage
      \global\@colroom\@colht
      \process@deferreds}%
   \mult@info\@ne
     {Colroom:\MessageBreak
      \the\@colht\space
              after float space removed
              = \the\@colroom \@gobble}%
    \set@mult@vsize \global
  \fi}

\makeatother
\setlength{\parindent}{0pt}

\begin{document}
\small

\begin{multicols*}{3}

    \section{Entrer ... et sortir}

    \subsection{invocation}

    \begin{tabular}{m{7cm} l}
        \texttt{vim file1 file2 ... fileN}   & \\
        \texttt{vim +linenumber file}  & \\
        \texttt{vim scp://login@serveur/path/to/file}  & \\
        \texttt{vim scp://login@serveur/path/}  & \\
    \end{tabular}

    \subsection{sortir}
    \begin{tabular}{m{2cm} l}
        \texttt{:q}   & Quitter\\
        \texttt{:wq}  & Sauvegarder et quitter \\
        \texttt{:q!}  & Quitter sans sauver\\
    \end{tabular}

    \section{Se déplacer}
    \begin{tabular}{m{2cm} l}
        \texttt{h,j,k,l}  & Se déplacer \\
    \end{tabular}
    \subsection{Dans la ligne}
    \begin{tabular}{m{2cm} l}
        \texttt{0}  & Aller au début de la ligne \\
        \texttt{\$} & Aller à la fin de la ligne \\
        \texttt{\^} & Aller au 1er char. de la ligne \\
        \texttt{w}  & Mot suivant\\
        \texttt{b}  & Mot suivant (précédé par un espace)\\
    \end{tabular}
    \subsection{Dans le fichier }
    \begin{tabular}{m{2cm} l}
        \texttt{gg} & Aller au début de la 1ère ligne \\
        \texttt{G}  & Aller au début de la dernière ligne \\
        \texttt{:n} & Aller à la ligne n \\
        \texttt{\%} & Aller à la parenthèse correspondante \\
    \end{tabular}

    \section{Passer en mode insertion}
    \begin{tabular}{m{2cm} l}
        \texttt{i}  & A l'emplacement du curseur\\
        \texttt{a}  & Après le curseur\\
        \texttt{A}  & A la fin de la ligne\\
        \texttt{o}  & En ouvrant une ligne en dessous\\
        \texttt{O}  & En ouvrant une ligne au dessus\\
    \end{tabular}

    \section{Passer en mode visuel}
    \begin{tabular}{m{2cm} l}
        \texttt{v}& Mode visuel charactère\\
        \texttt{V}& Mode visuel ligne\\
        \texttt{CTRL+v}& Mode visuel block\\
    \end{tabular}

    \section{Undo / Redo}
    \begin{tabular}{m{2cm} l}
        \texttt{u}& Undo\\
        \texttt{CTRL+r}& Redo\\
    \end{tabular}\\

    \section{ Sans quitter le mode insert}
    \begin{tabular}{m{2cm} l}
        \texttt{CTRL+w} & Efface le mot précédent\\
        \texttt{CTRL+u} & Efface depuis le début de la ligne\\
    \end{tabular}\\

    \section{Ouvrir / Sauver}
    \begin{tabular}{m{2cm} l}
        \texttt{:e fichier}& Ouvrir un fichier dans un buffer\\
        \texttt{:w}& Sauver\\
        \texttt{:w fichier}& Sauver Sous \\
    \end{tabular}\\
    \section{Gestion des buffers}
    \begin{tabular}{m{2cm} l}
        \texttt{:ls}& Lister les buffers\\
        \texttt{:b7}& Afficher le buffer 7\\
        \texttt{:bn}& Buffer suivant\\
        \texttt{:bp}& Buffer précédent\\
    \end{tabular}\\

    \section{Gestion des fenêtres}
    \begin{tabular}{m{2cm} l}
        \texttt{CTRL+w v}& Couper verticalement\\
        \texttt{CTRL+w s}& Couper horizontalement\\
        \texttt{CTRL+w $\leftarrow$ }& Aller dans la fenêtre à gauche\\
        \texttt{CTRL+w $\rightarrow$ }&Aller dans la fenêtre à droite\\
        \texttt{CTRL+w $\uparrow$ }&Aller dans la fenêtre au dessus\\
        \texttt{CTRL+w $\downarrow$ }&Aller dans la fenêtre en dessous\\
        \texttt{CTRL+w c }&Fermer la fenêtre\\
    \end{tabular}\\

    \section{Copier / Coller}
    \subsection{ Copier la sélection courante }
    \begin{tabular}{m{2cm} l}
        \texttt{y}& Copier\\
        \texttt{d}& Couper\\
    \end{tabular}
    \subsection{ Coller }
    \begin{tabular}{m{2cm} l}
        \texttt{p}& Coller (après)\\
        \texttt{P}& Coller (avant)\\
    \end{tabular}
    \subsection{ Copier sans selection }
    \begin{tabular}{m{2cm} l}
        \texttt{yy}& Copier la ligne courante\\
        \texttt{dd}& Couper la ligne courante\\
        \texttt{yny}& Copier n lignes \\
        \texttt{yw}& Copier le mot courant\\
        \texttt{y\$}& Copier du curseur à la fin de la ligne\\
    \end{tabular}

    \subsection{ Utilisation des registres }
    \begin{tabular}{m{2cm} l}
        \texttt{"ay}& Copier dans le registre a\\
        \texttt{"Ay}& Copier et ajouter au registre a\\
        \texttt{:reg}& Voir les registres et leur contenu\\
    \end{tabular} \\

    Les registres utilisables sont les suivants :\\

    \begin{tabular}{m{2cm} l}
        \texttt{a..z}& 26 registres généraux\\
        \texttt{1..9}& Registres des 9 derniers usages de d\\
        \texttt{+}   & Registre presse-papier système (ctrl+c)\\
        \texttt{*}   & Registre presse-papier système (souris)\\
        \texttt{\%}  & Registre contenant le nom du fichier\\
    \end{tabular} \\

    \section{Rechercher}
    \begin{tabular}{m{2cm} l}
        \texttt{/mot}& Recherche "mot"\\
        \texttt{n}& Suivant\\
        \texttt{N}& Précédent\\
    \end{tabular}

    \section{Remplacer}
    
    \texttt{:\%s{\char`\/}expr1/expr2/g} \\
    
    \begin{tabular}{m{2cm} l}
        \texttt{expr1 }   &  L'expression à remplacer\\
        \texttt{expr2 }   &  L'expression de remplacement \\
        \texttt{\% }   &  La portée de la recherche\\
        \texttt{g }   &  Options\\
    \end{tabular}\\

    \subsection{Portée:}
    \begin{tabular}{m{2cm} l}
        \texttt{n,m}  & De la ligne n à la ligne m \\
        \texttt{.,+5} & La ligne courante et les 5 lignes suivantes \\
        \texttt{\%}   & dans tout le fichier \\
    \end{tabular}\\

    \subsection{Options:}
    \begin{tabular}{m{2cm} l}
        \texttt{g} & Continuer après le premier match sur la ligne\\
        \texttt{c} & Demander confirmation\\
        \texttt{i} & Ignorer la casse\\
    \end{tabular}\\

    \section{Correction orthographique}
    \begin{tabular}{m{4cm} l}
        \texttt{:spell lang=fr} & Démarrer le correcteur\\
        \texttt{:setlocal spell lang=fr} & Correction dans le buffer courant\\
        \texttt{:set nospell} & Arrête le correcteur\\
        \texttt{]s} & Mot erroné suivant\\
        \texttt{[s} & Mot erroné précédent\\
        \texttt{z=} & Suggérer une correction\\
        \texttt{zg} & Mot correct (ajoute au dico)\\
        \texttt{zw} & Marque le mot comme incorrect\\
    \end{tabular}\\

    \section{Le coin du codeur}
    \subsection{Incrémenter / Décrémenter}
    \begin{tabular}{m{2cm} l}
        \texttt{CTRL+a} & Incrémenter un nombre\\
        \texttt{CTRL+x} & Décrémenter un nombre\\
    \end{tabular}\\
    \subsection{indentation}
    \begin{tabular}{m{2cm} l}
        \texttt{=}  & Indente auto. la selection\\
        \texttt{==} & Indente auto. la ligne\\
        \texttt{$>$}  & Augmente l'indentation \\
        \texttt{$<$}  & Réduit l'indentation \\
        \texttt{$>>$} & Augmente l'indentation de la ligne\\
        \texttt{$<<$} & Réduit l'indentation de la ligne\\
        \texttt{$n>>$} & Augmente de n indentations la ligne\\
    \end{tabular}\\
    \subsection{Compilation}
    \begin{tabular}{m{2cm} l}
        \texttt{:make}  & Lance make dans le repertoire courant\\
    \end{tabular}\\
    \subsection{Commandes Shell}
    \begin{tabular}{m{2cm} l}
        \texttt{:r fichier} & Insère le contenu de <fichier> \\
        \texttt{:! cmd}  & Execute la commande cmd dans un shell\\
        \texttt{:r! cmd} & Execute cmd et insère le résultat \\
    \end{tabular}\\

    \section{Gestion des sessions}
    \begin{tabular}{m{4cm} l}
        \texttt{:mks fichier.vim} & Enregistre la session\\
        \texttt{:source fichier.vim} & Restaure la session\\
    \end{tabular}\\
    
    \section{Gestion de fichiers (netrw) }
    \begin{tabular}{m{2cm} l}
        \texttt{:Explore} & Ouvre le gestionnaire de fichiers\\
    \end{tabular}\\
    \subsection{ Opérations}
    \begin{tabular}{m{2cm} l}
        \texttt{D} & Effacer fichier/repertoire\\
        \texttt{d} & Créer un repertoire\\
        \texttt{\%} & Ouvrir un nouveau fichier\\
    \end{tabular}\\
    \subsection{ Marquer / Démarquer}
    \begin{tabular}{m{2cm} l}
        \texttt{mf} & Marquer un fichier\\
        \texttt{mu} & Démarquer tous les fichiers\\
        \texttt{mp} & Afficher les fichiers marqués\\
        \texttt{mt} & Taguer le repertoire courant\\
    \end{tabular}\\
    \subsection{ Opérations sur les fichiers marqués}
    \begin{tabular}{m{2cm} l}
        \texttt{me} & Editer\\
        \texttt{md} & Diff\\
        \texttt{mc} & Copier dans le repertoire tagué\\
        \texttt{mm} & Déplacer dans le repertoire tagué\\
    \end{tabular}\\

\end{multicols*}
\newpage

\end{document}

